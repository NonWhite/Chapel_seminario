\section{Introdução}
Message Passing Interface(MPI) é um padrão para comunicação de dados para aplicações que requerem computações paralelas. O padrão define a estrutura e a funcionalidade de uma serie de rutinas que permitem a comunicação entre os processadores, nós de um cluster, etc. 

Neste padrão, uma aplicação é percevida como um ou mais processos que se comunicam mediante o acionamento de funções para o envio e recebimento de mensagens. Os processos podem usar mecanismos de comunicação ponto a ponto ou operações coletivas de comunicação (operações globais).

O Open MPI é uma implementação open source de MPI. Open MPI apresenta as seguintes caracteristicas:
\begin{itemize}
	\item Comformidade com MPI-3
	\item Concurrencia e seguridade de threads
	\item Tolerancia a erros nos processos e redes
	\item Soporte a diversos tipos de redes
	\item Soporte de mais de um agendador de trabalhos
	\item Soporte para diversos sistemas operacionais
	\item Portavel e mantenivel
	\item Disenho a base de componentes, APIs documentadas
	\item Licencia BSD 
\end{itemize}

\clearpage
